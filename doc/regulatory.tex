%% regulatory.tex
%% Copyright 2023 E. Nijenhuis
%
% This work may be distributed and/or modified under the
% conditions of the LaTeX Project Public License, either version 1.3
% of this license or (at your option) any later version.
% The latest version of this license is in
% http://www.latex-project.org/lppl.txt
% and version 1.3 or later is part of all distributions of LaTeX
% version 2005/12/01 or later.
%
% This work has the LPPL maintenance status ‘maintained’.
%
% The Current Maintainer of this work is E. Nijenhuis.
%
% This work consists of the files regulatory.tex, fc-dutch.def
% and regulatory.sty
\documentclass[10pt,dutch]{ltxdoc}

%! suppress = InclusionLoop
\usepackage[markdown]{regulatory}
\usepackage{tabularx}
\usepackage[dutch,english]{babel}
\usepackage{listings}

\usepackage{fontspec}
\usepackage{textcomp}

\definecolor{greycolor}{HTML}{BFBFBF}
\definecolor{primary}{HTML}{174A67}
\definecolor{darkcolor}{HTML}{091D28}
\definecolor{greencolor}{HTML}{176734}
\colorlet{artlabel}{primary}
\colorlet{green}{greencolor}

\zcRefTypeSetup{listing}{%
    Name-sg = Codelijst,
    name-sg = codelijst,
    Name-pl = Codelijsten,
    name-pl = codelijsten
}

\renewcommand{\lstlistingname}{Codelijst}

\lstdefinestyle{tex}{%
    language=[LaTeX]TeX,
    keywordsprefix={\\},
    alsoletter={\\},
    morekeywords={article,para,textfill,masterdocument,refdocument,loadglsdefs,printdefs,setmiddleconjunction,setlastconjunction,setrangeconjunction,setconjunction,markdownInput,glssetwidest,describe}
}

\lstdefinestyle{bash}{%
    language=bash,
    morekeywords={pdflatex,lualatex,latexmk,makeglossaries,bib2gls}
}

\lstdefinelanguage{Markdown}[LaTeX]{TeX}{%
    sensitive=false,
    morecomment=[l][\bfseries\ttfamily\color{green}]{\#},
    morecomment=[s][\bfseries\ttfamily\color{purple}]{:\ \ }{\ },
%    morecomment=[f][\bfseries\ttfamily\color{purple}]{:},
    morestring=[s][\color{black}]{\{}{\}},
    morekeywords={article,para,textfill,masterdocument,refdocument,loadglsdefs,printdefs,setmiddleconjunction,setlastconjunction,setrangeconjunction,setconjunction,markdownInput,glssetwidest,describe},
    keywordsprefix={\\},
    alsoletter={\\},
}

\lstdefinestyle{md}{%
    language=Markdown
}

\lstdefinelanguage{BibTeX}{%
    keywords={%
        article,book,collectedbook,conference,electronic,entry,ieeetranbstctl,%
        inbook,incollectedbook,incollection,injournal,inproceedings,%
        manual,mastersthesis,misc,patent,periodical,phdthesis,preamble,%
        proceedings,standard,string,techreport,unpublished%
    },
    keywordsprefix={@},
    comment=[l][\itshape]{@comment},
    sensitive=false,
}

\lstdefinestyle{bib}{%
    language=BibTeX,
    morestring=[s][\bfseries\ttfamily\color{purple}]{\ \ \ \ }{\ },
    morecomment=[n][\bfseries\ttfamily\color{green}]{\ \{}{\}},
}

\lstset{
    title=\lstname,
    basicstyle=\ttfamily,
    numberstyle=\ttfamily,
    numbersep=5pt,
    rulecolor=\color{greycolor},
%    fillcolor=\color{greycolor},
    stringstyle=\color{pink},
    keywordstyle=\color{primary},
%    keywordstyle=\bfseries\color{primary},
    commentstyle=\itshape\color{darkcolor},
    breaklines=true,
    captionpos=b,
    tabsize=4,
%    showspaces=true,               % show spaces adding particular underscores
%    showstringspaces=true,         % underline spaces within strings
    showtabs=true,
%    keepspaces=false
%    frameshape={RYRYNYYYY}{yny}{yny}{RYRYNYYYY}
%    framexleftmargin=5mm
}

\newcommand\package{\texttt}
\newcommand\file{\texttt}
\newcommand\option{\texttt}

\def\packagedate{27-03-2023}
\def\packageversion{0.0.2}

\masterdocument[ex1-]{example1}{%
    defs=example1,%
    author=E. Nijenhuis,%
    subject= Voorbeeld Één,%
    description = Het éérste voorbeeld document,
    ref label=van Voorbeeld Één
}

\masterdocument[ex2-]{example2}{%
    defs=example2,%
    author=E. Nijenhuis,%
    subject= Voorbeeld Twee,%
    description = Het tweede voorbeeld document,
    ref label=van Voorbeeld Twee
}

\setmainfont{Prompt ExtraLight}
\setmonofont{Jetbrains Mono}

\def\cmd{\lstinline[style=tex]}
\def\cmditem#1{\item[\ttfamily\textbackslash{}#1]}

\newcommand\Vtextvisiblespace[1][.3em]{%
  \mbox{\kern.06em\vrule height.3ex}%
  \vbox{\hrule width#1}%
  \hbox{\vrule height.3ex}}

% Document
\begin{document}
    \selectlanguage{dutch}
    \title{Het \package{regulatory} pakket\thanks{Dit document correspondeerd aan \textsf{regulatory}~\packageversion, geschreven op \packagedate.}}
    \author{Erik Nijenhuis\\
    \href{mailto:erik@xerdi.com}{erik@xerdi.com}}
    \date{\packagedate}
    \maketitle
    \begin{abstract}
        % Structuur %
        Het \package{regulatory} pakket leent zich uitstekend voor juristen in brede zin.
        Dit pakket brengt veel voorkomende structuren, zoals artikelen, leden, onderdelen en definities.

        % Verwijzen %
        Verwijzen binnen het juridisch domein kan een grote uitdaging zijn, daarom biedt dit pakket een elegante manier van verwijzen net als iemand mag verwachten bij standaard \LaTeX{} macro's, zoals voor een hoofdstuk \cmd{\section}, namelijk door middel van labelen.
        Hiervoor introduceert \package{regulatory} zijn eigen \cmd{\rref}, \cmd{\nref} en \cmd{\aref} macrofamilies die standaard ondersteuning bieden voor Nederlands en Engels.

        % Definities %
        Het beheren van definities kan eenduidig voor alle documenten met behulp van \package{BibTeX}.
        Voor het verwijzen naar definities speelt \package{regulatory} in op de al bestaande \cmd{\gls} macrofamilie van \package{glossaries}.

        % Externe Documenten %
        Het verwijzen naar artikelen, leden, onderdelen en definities beperkt zich niet alleen tot één document, maar zijn ook aan te halen vanuit andere documenten geschreven met het \package{regulatory} pakket.
        Op deze manier blijft het eenvoudig verwijzen voor documenten onderling.
        Denk b\ij{}voorbeeld aan Algemene Voorwaarden en een Onderhoudsovereenkomst die naar elkaars artikelen verwijzen of elkaars begrippen gebruiken.
        Het is zelfs mogelijk om bij documenten de Algemene Voorwaarden als bijlage in het PDF-bestand te voegen voor de volledigheid.
    \end{abstract}

    \clearpage
    \tableofcontents
    \clearpage

    \section{Gebruik}
    Het pakket \package{regulatory} is uitdrukkelijk bedoeld voor het genereren van PDF-documenten met \LaTeX{}.
    Gebruik daarom \texttt{pdflatex} of \texttt{lualatex}.
    \begin{lstlisting}[style=tex,caption={main.tex},label=code:simple]
\documentclass[dutch]{article}
\usepackage{regulatory}
\begin{document}
    \article{Mijn eerste artikel}
\end{document}
    \end{lstlisting}
    Het voorbeeld gebruikt geen opties.
    Dit houdt in dat \option{bib2gls} actief is.
    Om terug te schakelen naar TeX code voor definitielijsten is er de optie \oarg{nobibdefs}.
    Verder zijn er nog de opties \oarg{markdown,alldefs,hidelinks,nameinlink}.
    Waar \option{markdown} Markdown support activeert, \option{alldefs} alle definities opsomt in plaats van alleen de gebruikte definities binnen hetzelfde document (handig voor Algemene Voorwaarden waarin niet alle definities per se voorkomen), \option{hidelinks} alle gekleurde kaders van hyperlinks verbergt en \option{nameinlink} de hyperlink om het label heen plaatst.\\

    \noindent
    Het voorbeeld van \zcref{code:simple} kan als volgt gegenereerd worden naar PDF:
    \begin{lstlisting}[style=bash,caption={Commandline voorbeelden}]
pdflatex main
# Of
lualatex main
# Of blijvend genereren met LatexMK
latexmk -pvc -lualatex -interaction=nonstopmode main
    \end{lstlisting}

    \noindent
    Stel er worden definitie lijsten gebruikt, dan komen er nog extra stappen bij in het generatieproces, namelijk:
    \begin{lstlisting}[style=Bash,caption={Commandline met definities}]
lualatex main
bib2gls main
lualatex main
lualatex main
# Of voor TeX code
lualatex main
makeglossaries main
lualatex main
lualatex main
    \end{lstlisting}
    In het geval gebruik gemaakt wordt van \texttt{latexmk} kan er in een aparte terminal het commando \texttt{bib2gls} of \texttt{makeglossaries} gebruikt worden.
    LatexMK ziet vanzelf de bestanden wijzigen en genereert dan het document opnieuw.

    \clearpage

    \section{Structuur}\label{sec:struct}
    Dit pakket levert bekende structuren, zonder bestaande functionaliteiten te breken.
    \DescribeMacro{\article}
    \DescribeMacro{\para}
    Neem b\ij{}voorbeeld \cmd|\article|\marg{title} en \cmd|\para|\marg{title}.
    Deze zijn als aparte macro's gedefinieerd en opgemaakt met behulp van \package{titlesec}.\\

    \noindent
    \DescribeEnv{paras}
    Voor de \option{paras} en ``onderdelen'' is een nieuwe environment aangemaakt met behulp van \package{enumitem}.
    De labels van de leden zijn aangepast naar \cmd|\theartikel.\arabic*|, voor leden en, \cmd|\abalphnum{\arabic*)}|, voor onderdelen.
    Voor onderdelen wordt er gebruik gemaakt van \cmd|\abalpnum| uit pakket \package{fmtcount}\footnote{Het pakket \package{fmtcount} heeft geen ondersteuning voor Nederlands. Dit pakket levert daarvoor de juist configuratie mee.} om meerdere onderdelen te kunnen opsommen.
    Stel er zou gebruik gemaakt worden van \cmd|\alph|, dan is \option{paras} beperkt tot 26 onderdelen.
    Bij \cmd|\abalphnum| met b\ij{}voorbeeld een waarde van 125 is `\texttt{du}' de uitkomst.

    \begin{lstlisting}[style=tex]
\article{Voorbeeld}
\begin{paras}
   \item \textfill
   \begin{paras} % leden en onderdelen maken gebruik van dezelfde environment
     \item \textfill
   \end{paras}
   \item \textfill
\end{paras}

% Of leden met titels, zoals in een Privacyverklaring

\article{Voorbeeld2}
\textfill

\para{Voorbeeld3}
\textfill
    \end{lstlisting}
    Zie \zcref{code:example1,code:example2} voor meer \LaTeX{} voorbeelden.

%    \subsection{Vormgeving}
%    TODO

    \subsection{Markdown}
    Met de pakketoptie \option{markdown} zorgt dit pakket ervoor dat al deze structuren gehanteerd worden.
    Dit betekend echter wel dat er geen hoofdstukken of andere standaard onderdelen meer getypt kunnen worden.
    In plaats daarvan is de schrijver juist beperkt tot de onderdelen omschreven in dit hoofdstuk.
    Kijk naar \zcref{code:example-md} voor een markdown voorbeeld en naar \zcref{code:md-example} hoe zo'n Markdown bron uiteindelijk kan gebruikt worden vanuit \LaTeX{}.

    \clearpage
    \section{Definities}
    Voor het verwijzen naar definities wordt gebruik gemaakt van \package{glossaries-extra}.
    Dit zorgt ervoor dat er met de \cmd|\gls|\marg{label} macrofamilie verwezen kan worden.\\

    \noindent
    \DescribeEnv{definitions}
    \DescribeEnv{externals}
    Om conflicten tussen begrippen, afkortingen en definities te voorkomen voegt dit pakket twee \texttt{glossary} types toe.
    Voor definities binnen hetzelfde bestand worden ze onder het type \option{definitions} geplaatst, waar bij definities van andere documenten ze geplaatst worden onder \option{externals} (zie \zcref{sec:extern}).\\

    \noindent
    \DescribeMacro{\printdefs}
    Om definities op te sommen zijn er verschillende macro's toegevoegd.
    De meest voor de hand liggende, \cmd|\printdefs|\marg{width of text}, worden de definities met aangepaste stijl opgesomd.
    Het argument \marg{width of text} wordt meegegeven om wille van de uitlijning van definitie aanduidingen en beschrijvingen.\\

    \noindent
    \DescribeMacro{\describe}
    Om hetzelfde resultaat te behalen met \cmd|\describe|\marg{label} is het eerst nodig om \cmd|\glssetwidest| aan te roepen in b\ij{}voorbeeld Markdown (zie \zcref{code:example-md}).
    De macro \cmd|\describe| leent zich uitstekend om handmatig definitie beschrijvingen te plaatsen.
    Deze macro voegt namelijk een ankerpunt toe, vereist voor werkende hyperlinks.\\

    \noindent
    \DescribeMacro{\loadglsdefs}
    Om definities in te laden zijn er twee macro's gedefinieerd.
    De \cmd|\loadglsdefs|\marg{src} macro voegt BibTeX bibliotheken toe onder het type \option{definitions} en heeft als categorie \option{definitions}.
    Gebruikte definities in deze bibliotheken zullen opgesomd worden wanneer \cmd|\printdefs| wordt aangeroepen.
    Als bij het pakket de optie \option{alldefs} is meegegeven, dan zullen alle definities in die bibliotheken opgesomd worden.
    De opsomming wordt gesorteerd aan de hand van het Nederlandse woordenboek.
    Woorden die niet voorkomen daarin worden als eerste opgesomd.\\

    \noindent
    \DescribeMacro{\loadextdefs}
    Voor \cmd|\loadextdefs|\oarg{category}\marg{src} kan het handig zijn om een categorie mee te geven, zodat definities van verschillende bronnen uit elkaar gehouden kunnen worden.
    Het is daarentegen onverstandig deze direct aan te roepen, aangezien \cmd|\masterdocument| hier al slim op in speelt.

    \clearpage
    \section{Verwijzingen}

    Voor het verwijzen naar artikelen, leden en onderdelen wordt onder water gebruik gemaakt van \package{zref}.
    Alle onderdelen genoemd in \zcref{sec:struct} zijn hiervoor geconfigureerd, echter biedt \package{zref} niet zoveel formaat aanpassingen als \package{cleveref}.
    \DescribeMacro{\rref}
    \DescribeMacro{\Rref}
    Door deze beperking is er voor gekozen om geheel nieuwe varianten te ontwikkelen, waaronder \cmd|\rref|\marg{label}.
    Er kan dus met \cmd|\rref| verwezen worden naar artikelen net als gebruikelijk is voor \cmd|\section|, \cmd|\subsection|, en dergelijke.
    De \cmd|\rref| familie kent in totaal vier verschillende macro's:
    \begin{labeling}{\textbackslash{}Rref*\quad}
        \cmditem{rref} Beginnend met een kleine letter en met hyperlink.\\
        \begin{tabularx}{\linewidth}{@{}X X X@{}}
            \textbf{Artikel} & \textbf{Lid} & \textbf{Onderdeel}\\
            \rref{ex1-art:lorem} & \rref{ex1-lid:lorem} & \rref{ex1-sub:lorem}\\
        \end{tabularx}
        \cmditem{rref*} Beginnend met een kleine letter en zonder hyperlink.\\
        \begin{tabularx}{\linewidth}{@{}X X X@{}}
            \textbf{Artikel} & \textbf{Lid} & \textbf{Onderdeel}\\
            \rref*{ex1-art:lorem} & \rref*{ex1-lid:lorem} & \rref*{ex1-sub:lorem}\\
        \end{tabularx}
        \cmditem{Rref} Beginnend met een hoofdletter en met hyperlink.\\
        \begin{tabularx}{\linewidth}{@{}X X X@{}}
            \textbf{Artikel:} & \textbf{Lid:} & \textbf{Onderdeel:}\\
            \Rref{ex1-art:lorem} & \Rref{ex1-lid:lorem} & \Rref{ex1-sub:lorem}\\
        \end{tabularx}
        \cmditem{Rref*} Beginnend met een hoofdletter en zonder hyperlink.\\
        \begin{tabularx}{\linewidth}{@{}X X X@{}}
            \textbf{Artikel:} & \textbf{Lid:} & \textbf{Onderdeel:}\\
            \Rref*{ex1-art:lorem} & \Rref*{ex1-lid:lorem} & \Rref*{ex1-sub:lorem}\\
        \end{tabularx}
    \end{labeling}
    In de voorbeelden hierboven is al een opmerkelijk verschil te zien tussen alternatief \cmd|\zref|, namelijk de presentatie van de/het verwijsnummer/letter/woord en een afwijkend type in de titel.
    B\ij{}voorbeeld voor \texttt{ex1-lid:lorem} is de titel `2.1' en wordt aangehaald met `eerste'.\\

    \noindent
    \DescribeMacro{\nref}
    \DescribeMacro{\Nref}
    Om te verwijzen met de bijbehorende benaming is er de macrofamilie \cmd{\nref}\marg{label} ontwikkeld.
    Deze familie heeft net als \cmd{\rref} vier varianten.
    In het volgende voorbeeld gaan we voor het gemak alleen uit van \cmd{\Nref}.\\

    \noindent
    \begin{tabularx}{\linewidth}{@{}p{\widthof{EN}} X X X@{}}
        &\textbf{Artikel:} & \textbf{Lid:} & \textbf{Onderdeel:}\\
        NL & \Nref{ex1-art:lorem} & \Nref{ex1-lid:lorem} & \Nref{ex1-sub:lorem}\\
    \end{tabularx}\selectlanguage{english}\\
    \begin{tabularx}{\linewidth}{@{}p{\widthof{EN}} X X X@{}}
        EN & \Nref{ex1-art:lorem} & \Nref{ex1-lid:lorem} & \Nref{ex1-sub:lorem}\\
    \end{tabularx}\selectlanguage{dutch}\\

    \noindent
    Een opmerkelijk verschil tussen het alternatief \cmd|\zcref| is dat \cmd|\nref| rekening kan houden met de positie van de benaming.
    Kijk bijvoorbeeld naar de uitkomst van het lid ($\langle geschreven\ ordinaal\rangle\ \langle benoeming\rangle$).\\

    \noindent
    Met de macrofamilies \cmd|\rref| en \cmd|\nref| is er dus al veel mogelijk, echter zijn er nog veel andere factoren die meespelen als het gaat om verwijzen.
    Macro \cmd|\nref| doet bijvoorbeeld wel al de juiste benaming erbij, maar bij het verwijzen naar een lid wordt er geen bijbehorend artikel genoemd.
    \DescribeMacro{\aref}
    \DescribeMacro{\Aref}
    Voor volledige en automatische verwijzingen is de \cmd|\aref|\marg{labels...} ontwikkeld.
    Deze macrofamilie noteert dus alle onderdelen van de verwijzing.
    Daarnaast accepteert \cmd|\aref| meerdere labels en koppelt het de labels op de juiste manier.
    Dit kan een lijst opsomming geven, zoals \aref{ex1-sub:lorem,ex1-sub:lorem3,ex1-sub:lorem4}, of een reeks, zoals \aref{ex1-sub:lorem,ex1-sub:lorem2,ex1-sub:lorem3,ex1-sub:lorem4}.
    Hierop is alleen één tekortkoming, namelijk, de optie \option{nameinlink} kan niet toegepast worden wanneer er meerdere labels meegegeven worden.
    Deze tekortkoming geldt niet wanneer er één label wordt meegegeven.
    Een andere bijkomstigheid is dat deze verwijzingen gemakkelijk te vertalen zijn naar het Engels:\\
    \begin{lstlisting}[style=tex]
\selectlanguage{english}
See \aref{ex1-sub:lorem,ex1-sub:lorem3,ex1-sub:lorem4}
and \aref{ex1-sub:lorem,ex1-sub:lorem2,ex1-sub:lorem3,ex1-sub:lorem4}.
    \end{lstlisting}\selectlanguage{english}~\\
    \vspace*{-4em}
    \begin{center}
        \large
        See \aref{ex1-sub:lorem,ex1-sub:lorem3,ex1-sub:lorem4}
        and \aref{ex1-sub:lorem,ex1-sub:lorem2,ex1-sub:lorem3,ex1-sub:lorem4}.
    \end{center}
    \selectlanguage{dutch}

    \subsection{Conjunctie}
    \DescribeMacro{\setmiddleconjunction\\\marg{format}}
    \DescribeMacro{\setlastconjunction}
    \DescribeMacro{\setrangeconjunction}
    \DescribeMacro{\setconjunction\\\marg{middle}\\\marg{last}\\\marg{range}\\}
    Het koppelen van de labels gebeurt via \package{zref}.
    Er zijn echter dankzij de herimplementatie andere macro's geschreven op een vergelijkbare manier als \package{cleveref} dat doet, namelijk:
    \begin{lstlisting}[style=tex]
\setmiddleconjunction}{, }
\setlastconjunction}{ en }
\setrangeconjunction}{ t/m }
\setconjunction{, }{ en }{ t/m }
    \end{lstlisting}
    \DescribeMacro{\setjuncto}
    \DescribeMacro{\unsetjuncto}
    Er zijn ook macro's om gemakkelijk naar verouderde notatie te schakelen.
    Met \cmd|\setjuncto| kan er willekeurig in het document omgeschakeld worden naar het koppelwoord `\texttt{ jo.\textbackslash\ }'.
    Door \cmd|\unsetjuncto| te gebruiken wordt \cmd|\lastconjunction| weer terug gezet naar `\texttt{ en }'.
    Als eerder handmatig de koppelingen zijn veranderd, dan zijn die wijzigingen door één van deze macro's overschreven.
    Gebruik dan in plaats van \cmd|\unsetjuncto| \cmd|\setlastconjunction|\marg{waarde}.

    \clearpage
    \section{Verwijzingen naar andere documenten}\label{sec:extern}
    Verwijzen naar andere documenten gegenereerd met het \package{regulatory} pakket kan vrij eenvoudig.
    \DescribeMacro{\refdocument\\\oarg{prefix}\\\marg{name}\\\marg{opts...}}
    Na het opgeven van \cmd|\refdocument| in de preamble kan er naar artikelen, leden en onderdelen verwezen worden.
    De macro familie \cmd|\aref| blijft dankzij \cmd|\zexternaldocument| van \package{zref-xr} nog steeds werken.
    Om ambiguïteit te voorkomen kan er een speciale \option{prefix} toegevoegd worden.
    Bij het weglaten van een \option{prefix} is er alsnog een \option{prefix}, namelijk \texttt{ext-}.
    Label \texttt{lid:lorem} wordt dan \texttt{ext-lid:lorem}.
    Let wel dat de \option{prefix} niet geldt voor het verwijzen naar definities met \cmd|\gls|.\\

    \noindent
    \DescribeMacro{\masterdocument\\\oarg{prefix}\\\marg{name}\\\marg{opts...}\\}
    Om een volledige koppeling te maken met andere \package{regulatory} documenten wordt de \cmd|\masterdocument| macro gebruikt.
    Een volledige koppeling houdt in:
    \begin{enumerate}
        \item verwijzen met de \cmd|\aref| familie;
        \item verwijzen naar definities met de \cmd|\gls| familie;
        \item verwijzen naar het bijbehorende document;
        \item een voetnoot met het document als bijlage bij de eerste verschijning van een verwijzing of definitie.
    \end{enumerate}
    Het kan zelfs dat een document meerdere `master' documenten heeft, zoals dit document:
    \begin{lstlisting}[style=tex]
\newcommand\definitionlabel[1]{~(zie #1)}
\masterdocument[ex1-]{example1}{
    defs=example1,
    author=E. Nijenhuis,
    subject= Voorbeeld Één,
    description = Het éérste voorbeeld document,
    ref label=van Voorbeeld Één,
    def label=\definitionlabel
}

\masterdocument[ex2-]{example2}{
    defs=example2,
    author=E. Nijenhuis,
    subject= Voorbeeld Twee,
    description = Het tweede voorbeeld document,
    ref label=van Voorbeeld Twee,
    def label=\definitionlabel
}
    \end{lstlisting}

    Beide macro's \cmd{\refdocument} en \cmd{\masterdocument} hebben als derde argument keuze uit de volgende opties:
    \begin{labeling}{\texttt{footnote label}}
        \addtokomafont{labelinglabel}{\ttfamily}
        \item[name] standaard hetzelfde als het eerste argument van de macro's.
        \item[filename] standaard het eerste argument met \texttt{.pdf} geconcateneerd.
        Deze optie kan overschreven worden wanneer de naam niet correspondeerd met de bestandsnaam.
        \item[ref label] standaard heeft deze macro geen waarde en in dat geval wordt er in \cmd{\documentlabel} de volgende standaard waarde gebruikt: \lstinline|\GetTranslation{of the} \artifactsubject{#1}|.
        Deze macro krijgt als argument de \texttt{name} van het document.
        \item[def label] standaard heeft deze macro de volgende waarde: \lstinline|~(\GetTranslation{see} #1)|.
        Het argument bevat de \texttt{subject} met eventueel de footnote (afhankelijk van \texttt{referred}).
        \item[footnote] standaard is de waarde \texttt{true}, zodat er voetnoten geplaatst worden bij de eerste verschijning.
        Deze kan naar \texttt{false} gezet worden om dit te voorkomen.
        \item[footnote label] deze macro krijgt als argument het bijgevoegde document met \texttt{subject} als weergave in de tekst.
        Standaard print deze macro enkel het eerste argument.
        \item[url] deze optie wordt nog niet gebruikt.
        De bedoeling van deze optie is om in de voetnoot de vindplaats van het document te duiden.
        \item[referred] deze optie is voor intern gebruik.
        \cmd{\documentfootnote} zet deze waarde naar \texttt{true}.
        \item[defs] deze optie wordt gebruikt om externe definitielijsten in te laden onder dit document.
        Op deze manier kan bij eerste verschijningen van definities de juiste bron vermeldt worden.
        \item[author] deze optie wordt in \cmd|\documentattachment| gebruikt voor metadata t.b.v.\ sommige PDF-weergave applicaties.
        \item[subject] deze optie wordt net als \texttt{author} gebruikt in de bijlage.
        \item[description] deze optie wordt net als \texttt{author} gebruikt in de bijlage.
    \end{labeling}

    \noindent
    \DescribeMacro{\documentlabel\\\marg{label}}
    \DescribeMacro{\documentfootnote\\\oarg{link text}\\\marg{label}}
    \DescribeMacro{\documentattachment\\\marg{label}\\\marg{link text}}
    Voor zowel verwijzingen naar definities als artikelen e.d.\ wordt de bron vermeldt en wordt er bij de eerste verschijning een voetnoot geplaatst met een bijlage van het exemplaar.
    Onder water worden \cmd|\documentlabel|\footnote{Deze macro gebruikt het label van verwijzingen en dus niet van definities.} \cmd|\documentfootnote| en \cmd|\documentattachment| aangeroepen voor artikelen e.d.\ en definities.\\

    \noindent
    Deze macro's kunnen ook handmatig uitgevoerd worden.
    Neem b\ij{}voorbeeld \cmd|\documentfootnote{example2}| wat zou leiden tot:
    `\documentfootnote{example2}'.

    \clearpage
    \section{Taalondersteuning}
    In beginsel bood dit pakket alleen ondersteuning voor Nederlands.
    Toen Engels geïmplementeerd werd zijn er bepaalde macro's bij gekomen, zodat er makkelijk geschakeld kan worden tussen talen.
    De notatie van verwijzingen in het Engels en Nederlands verschillen der mate dat de instelling van een taal best complex kan zijn, echter maakt dit het wellicht instelbaar voor andere talen.\\

    \noindent
    \DescribeMacro{\rref@setup\\\marg{lang}\\\marg{article opts...}\\\marg{para opts...}\\\marg{sub opts...}}
    De \cmd|\rref@setup| macro heeft als eerste argument de taal die ingesteld dient te worden.
    De andere drie argumenten accepteren meerdere opties, namelijk:
    \begin{labeling}{\texttt{group conjunction}}
        \addtokomafont{labelinglabel}{\ttfamily}
        \item[name] de benaming in kleine letters.
        Bijvoorbeeld `artikel', `art.', `onderdeel', et cetera.
        Standaard voor \\artikel \lstinline|\GetTranslation{article}|, \\lid \lstinline|\GetTranslation{paragraph}| en \\onderdeel \lstinline|\GetTranslation{subparagraph}|.
        \item[Name] de benaming beginnend met een hoofdletter.
        Bijvoorbeeld `Artikel'.
        Net als \texttt{name} gebruikt deze optie standaard \lstinline|\GetTranslation|, maar dan met een hoofdletter.
        \item[ref format] een macro met één argument, namelijk het huidige nummer.
        Voor bijvoorbeeld een lid (of subparagraph in het Engels) is hiervoor \lstinline|\ordinalstringnum| gebruikt.
        In zo'n geval is het belangrijk om rekening te houden met hoofdletter gevoeligheid d.m.v.\ \lstinline|\@ifrrefcap|.
        Voorbeeld: \lstinline|\@ifrrefcap{\Ordinalstringnum{#1}}{\ordinalstringnum{#1}}|.
        \item[label format] de volgorde van de verwijzing en de naam.
        Deze macro krijgt twee argumenten mee, namelijk de \texttt{name} en het resultaat van \texttt{ref format}.
        Voor bijvoorbeeld een artikel is de volgorde \lstinline|{#1 #2}|, voor een lid \lstinline|{#2 #1}|, en voor een paragraph/lid in het engels \lstinline|{\@gobble{#1}#2}|\footnote{\lstinline|\\@gobble| verwerkt het argument maar print niks uit.}.
        Let wel dat je een macro dient op te geven.
        Bij het eerste voorbeeld zou je \cmd|\mylabelformat| als optie meegeven en zou gedefinieerd zijn als \cmd|\newcommand\mylabelformat[2]{#1 #2}|.
        \item[group conjunction] deze waarde geeft aan hoe het dient gekoppeld te worden aan het bovenliggende onderdeel.
        Bijvoorbeeld bij ``artikel\Vtextvisiblespace1\textcolor{red}{,\Vtextvisiblespace}eerste\Vtextvisiblespace lid'' is de \texttt{group conjunction} bij de instelling van het artikel gelijk aan \lstinline|{,~}|.
        \item[group format] deze optie accepteert een macro met één argument.
        Het eerste argument bevat alle onderliggende onderdelen.
        Bijvoorbeeld in het Engels dient bij een subparagraph/onderdeel het artikel en lid/paragraph omringd te worden met blokhaken.
        De waarde is dan bij het onderdeel \cmd|{{[}#1{]}~}|.
    \end{labeling}
    Deze opties kunnen verschillend zijn per artikel, lid en onderdeel.

    Bij het definiëren van een nieuwe taal is het belangrijk om te weten dat de standaard waarden gebaseerd zijn op de Engelse configuratie.
    Neem bijvoorbeeld de Nederlandse configuratie inclusief alle gebruikte hulp macro's:
    \begin{lstlisting}[style=TeX,caption={Nederlandse configuratie},numbers=left]
\newcommand\rref@refformat@noop[1]{#1}
\newcommand\rref@refformat@parenthesis[1]{(#1)}
\newcommand\rref@refformat@ordinal[1]{%
    \@ifrrefcap{%
        \Ordinalstringnum{#1}%
    }{%
        \ordinalstringnum{#1}%
    }%
}
\newcommand\rref@refformat@alpha[1]{%
    \@ifrrefcap{%
        \ABAlphnum{#1}%
    }{%
        \abalphnum{#1}%
    }%
}
\newcommand\rref@label@prefix[2]{#1 #2}
\newcommand\rref@label@postfix[2]{#2 #1}
\newcommand\rref@label@refonly[2]{\@gobble{#1}#2}
\newcommand\rref@group@braced[1]{{[}#1{]~}}

\rref@setup{dutch}{
    group conjunction={,~}
}{
    ref format=\rref@refformat@ordinal,
    label format=\rref@label@postfix,
    group conjunction={,~},
    group format=\rref@refformat@noop
}{
    ref format=\rref@refformat@alpha,
    label format=\rref@label@prefix,
    group conjunction={,~},
    group format=\rref@refformat@noop
}
    \end{lstlisting}

    \clearpage
    \section*{Voorbeelden}\label{sec:example}
    \lstset{numbers=left,frame=single}
    \lstinputlisting[style=tex,numbers=left,caption={example1.tex},label={code:example1}]{example1.tex}

    \lstinputlisting[style=tex,numbers=left,caption={example2.tex},label={code:example2}]{example2.tex}\clearpage

    \lstinputlisting[style=tex,numbers=left,caption={md-example.tex},label={code:md-example}]{md-example.tex}\clearpage

    \lstinputlisting[style=md,numbers=left,caption={example.md},label={code:example-md}]{example.md}\clearpage

    \lstinputlisting[style=bib,numbers=left,caption={example1.bib}]{example1.bib}

    \lstinputlisting[style=bib,numbers=left,caption={example2.bib}]{example2.bib}


\end{document}
