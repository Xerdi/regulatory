\documentclass[11pt,dutch]{ltxdoc}

%! suppress = InclusionLoop
\usepackage[markdown]{regulatory}

\usepackage{babel}
\usepackage{codelist}

\usepackage{fontspec}
\usepackage{textcomp}
\usepackage{listings}

\newcommand\package{\texttt}
\newcommand\file{\texttt}
\newcommand\option{\texttt}

\def\packagedate{17-03-2023}
\def\packageversion{0.0.1}

\definecolor{greycolor}{HTML}{BFBFBF}
\definecolor{brandcolor}{HTML}{174A67}
\definecolor{darkcolor}{HTML}{091D28}
\colorlet{artlabel}{brandcolor}

\masterdocument[ex1-]{example1}{%
    defs=example1,%
    author=E. Nijenhuis,%
    subject= Voorbeeld Één,%
    description = Het éérste voorbeeld document,
}

\masterdocument[ex2-]{example2}{%
    defs=example2,%
    author=E. Nijenhuis,%
    subject= Voorbeeld Twee,%
    description = Het tweede voorbeeld document,
}

% Document
\begin{document}
    \title{Het \package{regulatory} pakket\thanks{Dit document correspondeerd aan \textsf{regulatory}~\packageversion, geschreven op \packagedate.}}
    \author{Erik Nijenhuis\\
    \href{mailto:erik@xerdi.com}{erik@xerdi.com}}
    \date{\packagedate}
    \maketitle
    \begin{abstract}
        % Structuur %
        Het \package{regulatory} pakket leent zich uitstekend voor juristen in brede zin.
        Dit pakket brengt veel voorkomende structuren, zoals artikelen, leden, onderdelen en definities.

        % Verwijzen %
        Verwijzen binnen het juridisch domein kan een grote uitdaging zijn, daarom biedt dit pakket een elegante manier van verwijzen net als je gewend bent bij standaard \LaTeX{} macro's, zoals voor een \cmd{\section}, namelijk d.m.v.\ labelen.

        % Externe Documenten %
        Deze manier van verwijzen beperkt zich niet alleen tot één document, maar ook andere documenten geschreven met het \package{regulatory} pakket.
        Op deze manier blijft het eenvoudig verwijzen voor documenten onderling.
        Denk bijvoorbeeld aan Algemene Voorwaarden en een Onderhoudsovereenkomst die naar elkaars artikelen verwijzen.

        % Definities %
        Definities, typisch opgenomen in de Algemene Voorwaarden, zijn voor alle documenten gemakkelijk naar te verwijzen, dankzij \package{glossaries-extra}.
        Het is zelfs mogelijk om bij documenten de Algemene Voorwaarden als bijlage in het PDF-bestand te voegen voor de volledigheid.
    \end{abstract}

    \clearpage
    \tableofcontents
    \clearpage

    \section{Gebruik}
    Het pakket \package{regulatory} is uitdrukkelijk bedoeld voor het genereren van PDF-document met \LaTeX{}.
    Gebruik daarom \texttt{pdflatex} of \texttt{lualatex}.

    \begin{lstlisting}[style=TeX,caption={main.tex},label=code:simple]
\documentclass[dutch]{article}
\usepackage{regulatory}
\begin{document}
    \artikel{Mijn eerste artikel}
\end{document}
    \end{lstlisting}
    Het voorbeeld gebruikt geen opties.
    Dit houdt in dat \option{bib2gls} actief is.
    Om terug te schakelen naar TeX code voor definitielijsten is er de optie \oarg{nobibdefs}.
    Verder zijn er nog de opties \oarg{markdown,alldefs,hidelinks}.
    Waar \option{markdown} Markdown support activeert, \option{alldefs} alle definities opsomt (handig voor Algemene Voorwaarden waarin niet alle definities voorkomen), \option{hidelinks} alle gekleurde kaders van hyperlinks verbergt.\\

    Het voorbeeld van \zcref{code:simple} kan als volgt gegenereerd worden naar PDF:
    \begin{lstlisting}[style=Bash,caption={Commandline voorbeeld}]
pdflatex main
# Of
lualatex main
# Of blijvend genereren met LatexMK
latexmk -pvc -lualatex -interaction=nonstopmode main
    \end{lstlisting}

    Stel er worden definitie lijsten gebruikt, dan komen er nog extra stappen bij in het generatieproces, namelijk:
    \begin{lstlisting}[style=Bash,caption={Commandline met definities}]
lualatex main
bib2gls main
lualatex main
lualatex main
# Of voor TeX code
lualatex main
makeglossaries main
lualatex main
lualatex main
    \end{lstlisting}
    In het geval gebruik gemaakt wordt van \texttt{latexmk} kan er in een aparte terminal het commando \texttt{bib2gls} of \texttt{makeglossaries} gebruikt worden.
    LatexMK ziet vanzelf de bestanden wijzigen en genereert dan het document opnieuw.

    \clearpage

    \section{Structuur}\label{sec:struct}
    Dit pakket levert bekende structuren, zonder bestaande functionaliteiten te breken.
    \DescribeMacro{\artikel\marg{title}}
    \DescribeMacro{\lid\marg{title}}
    Neem bijvoorbeeld \cmd{\artikel\marg{title}} en \cmd{\lid\marg{title}}.
    Deze zijn als aparte macro's gedefinieerd en opgemaakt met behulp van \package{titlesec}.

    \DescribeEnv{leden}
    Voor de \option{leden} en ``onderdelen'' is een nieuwe environment aangemaakt met behulp van \package{enumitem}.
    De labels van de leden zijn aangepast naar \verb|\theartikel.\arabic*|, voor leden en, \verb|\abalphnum{\arabic*})|, voor onderdelen.
    Voor onderdelen wordt er gebruik gemaakt van \cmd{\abalpnum} uit pakket \package{fmtcount} om meerdere onderdelen te kunnen opsommen.
    Stel er zou gebruik gemaakt worden van \cmd{\alph}, dan is \option{leden} beperkt tot 26 onderdelen.
    Bij \cmd{\abalphnum} met bijvoorbeeld een waarde van 125 is `\texttt{du}' de uitkomst.

    \noindent\textbf{Voorbeeld:}
    \begin{lstlisting}[style=TeX]
\artikel{Voorbeeld}
\begin{leden}
   \item \tekstvulling
   \begin{leden} % leden en onderdelen maken gebruik van dezelfde environment
     \item \tekstvulling
   \end{leden}
   \item \tekstvulling
\end{leden}

% Of leden met titels, zoals in een Privacyverklaring

\artikel{Voorbeeld2}
\tekstvulling

\lid{Voorbeeld3}
\tekstvulling
    \end{lstlisting}
    Zie \zcref{code:example1,code:example2} voor meer \LaTeX{} voorbeelden.

%    \subsection{Vormgeving}
%    TODO

    \subsection{Markdown}
%    \lstinputlisting[style=Markdown]{example.md}

    \clearpage
    \section{Definities}
    Voor het verwijzen naar definities wordt gebruik gemaakt van \package{glossaries-extra}.
    Dit zorgt ervoor dat er met de \cmd{\gls\marg{label}} marco familie verwezen kan worden.\\

    \DescribeEnv{definitions}
    \DescribeEnv{externals}
    Om conflicten tussen begrippen, afkortingen en definities te voorkomen voegt dit pakket twee \texttt{glossary} types toe.
    Voor definities binnen hetzelfde bestand worden ze onder het type \option{definitions} geplaatst, waar bij definities van andere documenten ze geplaatst worden onder \option{externals} (zie \zcref{sec:extern}).\\

    \DescribeMacro{\uitlijning\marg{width of text}}
    \DescribeMacro{\describe\marg{label}}
    \DescribeMacro{\printdefs\marg{width of text}\\}
    Om definities op te sommen zijn er verschillende macro's toegevoegd.
    De meest voor de hand liggende, \cmd{\printdefs}, worden de definities met aangepaste stijl opgesomd.
    Het argument \marg{width of text} wordt meegegeven om wille van de uitlijning van definitie aanduidingen en beschrijvingen.

    De macro \cmd{\uitlijning} is gelijkwaardig aan \cmd{\glssetwidest} en is alleen nodig als de definitielijst handmatig wordt gespecificeerd in bijvoorbeeld Markdown formaat.
    De macro \cmd{\describe} leent zich uitstekend om handmatig definitie beschrijvingen te plaatsen.
    Deze macro voegt namelijk een ankerpunt toe, vereist voor werkende hyperlinks.\\

    \DescribeMacro{\loadglsdefs\marg{src}}
    \DescribeMacro{\loadextdefs\oarg{category}\\\marg{src}}
    Om definities in te laden zijn er twee macro's gedefinieerd.
    De \cmd{\loadglsdefs} macro voegt BibTeX bibliotheken toe onder het type \option{definitions} en heeft als categorie \option{definitions}.
    Gebruikte definities in deze bibliotheken zullen opgesomd worden wanneer \cmd{\printdefs} wordt aangeroepen.
    Als bij het pakket de optie \option{alldefs} is meegegeven, dan zullen alle definities in die bibliotheken opgesomd worden.
    De opsomming wordt gesorteerd aan de hand van het Nederlandse woordenboek.
    Woorden die niet voorkomen daarin worden als eerste opgesomd.

    Voor \cmd{\loadextdefs} kan het handig zijn om een categorie mee te geven, zodat definities van verschillende bronnen uit elkaar gehouden kunnen worden.
    Het is daarentegen onverstandig deze direct aan te roepen, aangezien \cmd{\masterdocument} hier al slim op in speelt.

    \clearpage
    \section{Verwijzingen}

    Voor het verwijzen naar artikelen, leden en onderdelen wordt onder water gebruik gemaakt van \package{zref}.
    Alle onderdelen genoemd in \zcref{sec:struct} zijn hiervoor geconfigureerd, echter biedt \package{zref} niet zoveel formaat aanpassingen als \package{cleveref}.
    \DescribeMacro{\aref\marg{labels...}}
    \DescribeMacro{\Aref\marg{labels...}}
    Door deze beperking is er voor gekozen om een geheel nieuwe variant te ontwikkelen, namelijk \cmd{\aref}.
    U kunt dus met \cmd{\aref} verwijzen naar artikelen net als gebruikelijk is voor \cmd{\section}, \cmd{\subsection}, e.d.

    \cmd{\Aref} kan zelfs meerdere labels meekrijgen gescheiden met een komma.
    Afhankelijk van de volgordelijkheid van de meegegeven labels koppelt \cmd{\aref} de labels op de juiste manier.
    Dit kan een lijst opsomming geven, zoals \aref{ex1-sub:lorem,ex1-sub:lorem3,ex1-sub:lorem4}, of een reeks, zoals
    \aref{ex1-sub:lorem,ex1-sub:lorem2,ex1-sub:lorem3,ex1-sub:lorem4}.

    % TODO: formaten %
    % \arabic*
    % \abalphnum{\arabic*}
    % \ordinalstringnum

    Een andere tekortkoming van \package{cleveref} is dat het niet automatisch de bijbehorende onderdelen kan vinden.
    Met \package{zref} kan dit wel geïmplementeerd worden, echter maakt alleen \cmd{\aref} hier gebruik van.
    Bijvoorbeeld \verb|\zref{sub:eerste}| resulteert in \zcref{ex1-lid:lorem}, in plaats van \aref{ex1-lid:lorem}.\\

    Deze macro's zijn geschreven met behulp van \package{zref} en \package{xassoccnt}.
    \cmd{\Aref} kijkt naar het eerste label en genereert daarna het voorvoegsel, bijvoorbeeld voor een lid is het voorvoegsel `artikel 1,~'.
    Daarna worden alle labels meegegeven aan een interne macro \cmd{\aref@group}.
    Wanneer er verschillende type labels meegegeven worden aan \cmd{\aref} is de uitkomst daarvan ongedefinieerd.
    Neem bijvoorbeeld \verb|\aref{lid:lorem,sub:lorem}|, wat in dit geval leidt tot een foutmelding.
    Dit is te verwachten, aangezien \texttt{lid:lorem} geen onderdeel uitmaakt van \texttt{art:cookies}.

    Het verschil met \cmd{\zref} is dat het een volledige verwijzing geeft en het juiste formaat hanteert.
    \verb|\aref{ex2-sub:lorem}| resulteert wel in `\aref{ex2-sub:lorem}', in plaats van alleen `\zcref{ex2-sub:lorem}'.\\

%    \DescribeMacro{\crefname}
%    Voor zowel \cmd{\zref} als \cmd{\aref} kunnen de aanduidingen van verwijzingen dankzij \package{cleveref} aangepast worden.
%    Stel dat `onderdeel a' een andere aanduiding dient te hebben, bijvoorbeeld `sub.\ a'.
%    Dit kan door middel van \cmd{\crefname}.
%    Deze macro krijgt drie argumenten, namelijk de \texttt{counter}, enkelvoud vorm en meervoudsvorm.
%    \begin{lstlisting}[style=TeX]
%\crefname{ledenii}{sub.\ }{sub.\ }
%    \end{lstlisting}
%    In het voorbeeld wordt `sub.' voor zowel enkel- als meervoud gebruikt.
%    De volgende counters kunnen gebruikt worden: \texttt{artikel, lid, ledeni, ledenii}.

    \subsection{Conjunctie}
    \DescribeMacro{\setmiddleconjunction\\\marg{format}}
    \DescribeMacro{\setlastconjunction}
    \DescribeMacro{\setrangeconjunction}
    \DescribeMacro{\setconjunction\\\marg{middle}\\\marg{last}\\\marg{range}\\}
    Het koppelen van de labels gebeurt via \package{zref}.
    Er zijn echter dankzij de herimplementatie andere macro's geschreven op een vergelijkbare manier als \package{cleveref} dat doet, namelijk:
    \begin{lstlisting}[style=TeX]
\setmiddleconjunction}{, }
\setlastconjunction}{ en }
\setrangeconjunction}{ t/m }
\setconjunction{, }{ en }{ t/m }
    \end{lstlisting}

    \DescribeMacro{\setjuncto}
    \DescribeMacro{\unsetjuncto}
    Er zijn ook macro's om gemakkelijk naar verouderde notatie te schakelen.
    Met \cmd{\setjuncto} kan er willekeurig in het document omgeschakeld worden naar het koppelwoord `\texttt{ jo.\textbackslash\ }'.
    Door \cmd{\unsetjuncto} te gebruiken wordt \cmd{\lastconjunction} weer terug gezet naar `\texttt{ en }'.
    Als eerder handmatig de koppelingen zijn veranderd, dan zijn die wijzigingen door één van deze macro's overschreven.
    Gebruik dan in plaats van \cmd{\unsetjuncto} \cmd{\setlastconjunction\marg{waarde}}.

    \section{Verwijzingen naar andere documenten}\label{sec:extern}
    Verwijzen naar andere documenten gegenereerd met het \package{regulatory} pakket kan vrij eenvoudig.

    \DescribeMacro{\refdocument\\\oarg{prefix}\\\marg{name}}
    Na het opgeven van \cmd{\refdocument} in de preamble kan er naar artikelen, leden en onderdelen verwezen worden.
    De macro familie \cmd{\aref} blijft dankzij \cmd{\zexternaldocument} van \package{zref-xr} nog steeds werken.
    Om ambiguïteit te voorkomen kan er een speciale \option{prefix} toegevoegd worden.
    Bij het weglaten van een \option{prefix} is er alsnog een \option{prefix}, namelijk \texttt{ext-}.
    Label \texttt{lid:geheimhouding} wordt dan \texttt{ext-lid:geheimhouding}.
    Let wel dat de \option{prefix} niet geldt voor het verwijzen naar definities met \cmd{\gls}.

    \DescribeMacro{\masterdocument\\\oarg{prefix}\\\marg{name}\\\marg{opts...}\\}
    Om een volledige koppeling te maken met andere \package{regulatory} documenten wordt de \cmd{\masterdocument} macro gebruikt.
    Een volledige koppeling houdt in:
    \begin{enumerate}
        \item verwijzen met de \cmd{\aref} familie;
        \item verwijzen naar definities met de \cmd{\gls} familie;
        \item een voetnoot met het document als bijlage bij de eerste verschijning van een verwijzing of definitie.
    \end{enumerate}
    Het kan zelfs dat een document meerdere `master' documenten heeft, zoals dit document:
    \begin{lstlisting}[style=TeX]
\masterdocument[ex1-]{example1}{
    defs=example1,
    author=E. Nijenhuis,
    subject= Voorbeeld Één,
    description = Het éérste voorbeeld document,
}

\masterdocument[ex2-]{example2}{
    defs=extra,
    author=E. Nijenhuis,
    subject= Voorbeeld Twee,
    description = Het tweede voorbeeld document,
}
    \end{lstlisting}


%    \DescribeMacro{\newartifact\marg{...}\\}
%
%    \DescribeMacro{\artifactname\\\marg{artifact}}
%    \DescribeMacro{\artifactfilename}
%    \DescribeMacro{\artifacturl}
%    \DescribeMacro{\artifactreferred}
%    \DescribeMacro{\artifactdefs}
%    \DescribeMacro{\artifactauthor}
%    \DescribeMacro{\artifactsubject}
%    \DescribeMacro{\artifactdescription\\}

    \clearpage

    \DescribeMacro{\reffootnotelabel\\\marg{label}}
    \DescribeMacro{\deffootnotelabel\\\marg{label}\\}
    Voor zowel verwijzingen naar definities als artikelen e.d.\ wordt bij de eerste verschijning een voetnoot geplaatst.
    Onder water wordt \cmd{\reffootnotelabel} aangeroepen voor artikelen e.d.\ en voor definities \cmd{\deffootnotelabel}.\\

    Deze macro's kunnen ook handmatig uitgevoerd worden.
    Neem bijvoorbeeld \verb|\aref{ex1-sub:lorem}\reffootnotelabel{example1}| wat zou leiden tot:
    `\aref{ex1-sub:lorem}\reffootnotelabel{example1}'.
    En voor \verb|\gls{def4}\deffootnotelabel{example2}| is de uitkomst `\gls{def4}\deffootnotelabel{example2}'.

%    \clearpage
%
%    \section{Entiteiten}\label{sec:entities}
%    \DescribeMacro{\newentity\marg{...}}
%    \blindtext
%
%    \DescribeMacro{\nameOf\marg{entity}}
%    \DescribeMacro{\titleOf\marg{entity}}
%    \DescribeMacro{\companyOf\marg{entity}}
%    \DescribeMacro{\kvkOf\marg{entity}}
%    \DescribeMacro{\websiteOf\marg{entity}}
%    \DescribeMacro{\emailOf\marg{entity}}
%    \DescribeMacro{\addressOf\marg{entity}}
%    \DescribeMacro{\postalOf\marg{entity}}
%    \DescribeMacro{\placeOf\marg{entity}}
%    \DescribeMacro{\ibanOf\marg{entity}}
%    \DescribeMacro{\signdateOf\marg{entity}}
%    \DescribeMacro{\signplaceOf\marg{entity}}
%    \blindtext
%
%    \subsection{Aanhef}
%    \DescribeMacro{\salute\marg{entity}}
%    \DescribeMacro{\salute*\marg{entity}}
%    \DescribeMacro{\Salute\marg{entity}}
%    \DescribeMacro{\Salute*\marg{entity}}
%    Om entiteiten aan te halen in tekst kan er gebruik gemaakt worden van de \cmd{\salute\marg{entiteit}} macro.
%    De macro kent in totaal vier verschillende varianten.
%    Voor een aanhaling beginnend met een hoofdletter is er \cmd{\Salute} en beginnend met kleine letter \cmd{\salute}.
%    Voor een afgekorte titel is een asterisk/\textasteriskcentered~nodig (\cmd{\salute*}).\\
%
%    \noindent\textbf{Voorbeeld:}
%    \begin{lstlisting}[style=TeX]
%\newentity{opdrachtnemer}{name = E. Nijenhuis}
%\newentity{opdrachtgever}{name = A. Richt, gender = female}
%\Salute{opdrachtnemer} en \salute*{opdrachtgever}
%    \end{lstlisting}
%    \newentity{opdrachtnemer}{name = E. Nijenhuis}
%    \newentity{opdrachtgever}{name = A. Richt, gender = female}
%    \textbf{Resultaat:}
%
%    \Salute{opdrachtnemer} en \salute*{opdrachtgever}
%
%    \vfill
%
%    \subsection{Hantekeningen}
%    \DescribeMacro{\signature\marg{entity}}
%%    Voor handtekening velden is er één macro — \verb|\signature{<entiteit>}| — die een entiteit als argument verwacht.
%%    \verb|\signature{<entiteit>}|\\
%
%    \noindent
%    \begin{minipage}{.49\linewidth}
%        \signature{contractor}
%    \end{minipage}%
%    \hfill
%    \begin{minipage}{.49\linewidth}
%        \signature{client}
%    \end{minipage}\\
%
%    \vfill
%
%    \subsection{Contactinformatie}
%    \DescribeMacro{\contact\marg{entiteit}}
%    \contact{contractor}

    \clearpage
    \section{Voorbeelden}\label{sec:example}

    \lstinputlisting[style=TeX,numbers=left,caption={example1.tex},label={code:example1}]{example1.tex}\clearpage

    \lstinputlisting[style=TeX,numbers=left,caption={example2.tex},label={code:example2}]{example2.tex}\clearpage

    \lstinputlisting[style=TeX,numbers=left,caption={md-example.tex},label={code:md-example}]{md-example.tex}

    \lstinputlisting[style=Markdown,numbers=left,caption={example.md},label={code:example-md}]{example.md}\clearpage

    \lstinputlisting[style=BiBTeX,numbers=left,caption={example1.bib}]{example1.bib}

    \lstinputlisting[style=BiBTeX,numbers=left,caption={example2.bib}]{example2.bib}


\end{document}
